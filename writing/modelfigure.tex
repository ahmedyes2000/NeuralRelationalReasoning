
\newcommand{\inputcolor}{black!20}
\newcommand{\outputcolor}{red!40}
\newcommand{\hiddencolor}{blue!40}
\newcommand{\hiddencolortwo}{orange!40}
\newcommand{\colorlabel}{green!40}
\newcommand{\compare}{cosine}




\begin{figure*}[t!]
  \begin{subfigure}[t]{0.25\textwidth}
  \centering
      \resizebox{92pt}{!}{%
\begin{tikzpicture}[
    % GLOBAL CFG
    font=\sf \large,
    >=LaTeX,
    % Styles
    rep/.style={% For representations
        rectangle, 
        rounded corners=3mm, 
        draw,
        very thick,
        minimum height =1cm, 
        minimum width=1.61cm
        },
    function/.style={%For functions
        ellipse,
        draw,
        inner sep=1pt
        },
    gt/.style={% For internal inputs
        rectangle,
        draw,
        minimum width=5mm,
        minimum height=4mm,
        inner sep=1pt
        },
    function/.style={
        rectangle,
        draw,
        minimum width=5mm,
        minimum height=4mm,
        inner sep=1pt
        },
    arrowconcat/.style={% Arrows for concatenation
        rounded corners=.25cm,
        dashed,
        thick,
        ->, 
        },
    arrowfunction/.style={% Arrows for concatenation
        rounded corners=.25cm,
        thick,
        ->,       
        }
    ]

%Start drawing the thing...    
    % Draw the cell: 
    \node [rep, fill=\inputcolor] (input1) at (-1.5,1.5){$x_1$} ;
    \node [rep, fill=\inputcolor] (input2) at (1.5,1.5){$x_2$} ;
    \node [rep, fill=\inputcolor] (concat) at (0,3){$x_1;x_2$} ;
    \draw [arrowconcat] (input1) -- (concat);
    \draw [arrowconcat] (input2) -- (concat);  
    
    
    \node [rep, fill=\hiddencolor] (hidden) at (0,5){$h$} ;
    \node [gt, minimum width=1cm] (relu) at (0,4-0.1) {$\ReLU$};
    \draw [arrowfunction] (concat) -- (relu) -- (hidden); 
    
    
    \node [rep,fill=\outputcolor] (output) at (0,7){$y$} ;
    \node [gt, minimum width=1cm] (softmax) at (0,6-0.1) {$\softmax$};
    \draw [arrowfunction] (hidden) -- (softmax) -- (output); 

\end{tikzpicture}
}
    \caption{A single layer network computing equality.}
    \label{fig:models:equality}
  \end{subfigure}
  \begin{subfigure}[t]{0.35\textwidth}
    \centering
      \resizebox{150pt}{!}{%
\begin{tikzpicture}[
    % GLOBAL CFG
    font=\sf \large,
    >=LaTeX,
    % Styles
    rep/.style={% For representations
        rectangle, 
        rounded corners=3mm, 
        draw,
        very thick,
        minimum height =1cm, 
        minimum width=1.61cm
        },
    function/.style={%For functions
        ellipse,
        draw,
        inner sep=1pt
        },
    gt/.style={% For internal inputs
        rectangle,
        draw,
        minimum width=5mm,
        minimum height=4mm,
        inner sep=1pt
        },
    function/.style={
        rectangle,
        draw,
        minimum width=5mm,
        minimum height=4mm,
        inner sep=1pt
        },
    arrowconcat/.style={% Arrows for concatenation
        rounded corners=.25cm,
        dashed,
        thick,
        ->, 
        },
    arrowfunction/.style={% Arrows for concatenation
        rounded corners=.25cm,
        thick,
        ->,       
        }
    ]

%Start drawing the thing...    
    % Draw the cell: 
    \node [rep, fill=\inputcolor] (input1) at (-3,1.5){$x_1$} ;
    \node [rep, fill=\inputcolor] (input2) at (-1,1.5){$x_2$} ;
    \node [rep, fill=\inputcolor] (input3) at (1,1.5){$x_3$} ;
    \node [rep, fill=\inputcolor] (input4) at (3,1.5){$x_4$} ;
    \node [rep, fill=\inputcolor] (concat) at (0,3){$x_1;x_2;x_3;x_4$} ;
    \draw [arrowconcat] (input1) -- (concat);
    \draw [arrowconcat] (input2) -- (concat);
    \draw [arrowconcat] (input3) -- (concat);
    \draw [arrowconcat] (input4) -- (concat);  
    
    
    \node [rep, fill=\hiddencolor] (hidden) at (0,5){$h_1$} ;
    \node [gt, minimum width=1cm] (relu) at (0,4-0.1) {$\ReLU$};
    \draw [arrowfunction] (concat) -- (relu) -- (hidden); 
    
    \node [rep, fill=\hiddencolortwo] (hidden2) at (0,7){$h_2$} ;
    \node [gt, minimum width=1cm] (relu2) at (0,6-0.1) {$\ReLU$};
    \draw [arrowfunction] (hidden) -- (relu2) -- (hidden2); 
    
    
    \node [rep,fill=\outputcolor] (output) at (0,9){$y$} ;
    \node [gt, minimum width=1cm] (softmax) at (0,8-0.1) {$\softmax$};
    \draw [arrowfunction] (hidden2) -- (softmax) -- (output); 

\end{tikzpicture}
}
    \caption{A two layer network computing hierarchical equality.}
    \label{fig:models:premack-deep}
  \end{subfigure}
  \begin{subfigure}[t]{0.4\textwidth}
    \centering
      \resizebox{200pt}{!}{%
\begin{tikzpicture}[
    % GLOBAL CFG
    font=\sf \large,
    >=LaTeX,
    % Styles
    rep/.style={% For representations
        rectangle, 
        rounded corners=3mm, 
        draw,
        very thick,
        minimum height =1cm, 
        minimum width=1.61cm
        },
    function/.style={%For functions
        ellipse,
        draw,
        inner sep=1pt
        },
    gt/.style={% For internal inputs
        rectangle,
        draw,
        minimum width=5mm,
        minimum height=4mm,
        inner sep=1pt
        },
    function/.style={
        rectangle,
        draw,
        minimum width=5mm,
        minimum height=4mm,
        inner sep=1pt
        },
    arrowconcat/.style={% Arrows for concatenation
        rounded corners=.25cm,
        dashed,
        thick,
        ->, 
        },
    arrowfunction/.style={% Arrows for concatenation
        rounded corners=.25cm,
        thick,
        ->,       
        }
    ]

%Start drawing the thing...    
    % Draw the cell: 
    \node [rep, fill=\inputcolor] (input1) at (-4.5,1.5){$x_1$} ;
    \node [rep, fill=\inputcolor] (input2) at (-1.5,1.5){$x_2$} ;
    \node [rep, fill=\inputcolor] (concat) at (-3,3){$x_1;x_2$} ;
    \draw [arrowconcat] (input1) -- (concat);
    \draw [arrowconcat] (input2) -- (concat);  
    
    
    \node [rep, fill=\hiddencolor] (hidden) at (-3,5){$h_1$} ;
    \node [gt, minimum width=1cm] (relu1) at (-3,4-0.1) {$\ReLU$};
    \draw [arrowfunction] (concat) -- (relu1) -- (hidden);
    
    
    \node [rep, fill=\inputcolor] (input3) at (4.5,1.5){$x_3$} ;
    \node [rep, fill=\inputcolor] (input4) at (1.5,1.5){$x_4$} ;
    \node [rep, fill=\inputcolor] (concat2) at (3,3){$x_3;x_4$} ;
    \draw [arrowconcat] (input3) -- (concat2);
    \draw [arrowconcat] (input4) -- (concat2);  
    
    
    \node [rep, fill=\hiddencolor] (hidden2) at (3,5){$h_2$} ;
    \node [gt, minimum width=1cm] (relu2) at (3,4-0.1) {$\ReLU$};
    \draw [arrowfunction] (concat2) -- (relu2) -- (hidden2);
    
    
    \node [rep,fill=\hiddencolor] (hiddenconcat) at (0,6){$h_1;h_2$};
    \draw [arrowconcat] (hidden2) -- (hiddenconcat);
    \draw [arrowconcat] (hidden) -- (hiddenconcat);
    
    
    \node [rep, fill=\hiddencolor] (hidden3) at (0,8){$h_3$} ;
    \node [gt, minimum width=1cm] (relu3) at (0,7-0.1) {$\ReLU$};
    \draw [arrowfunction] (hiddenconcat) -- (relu3) -- (hidden3);
    
    
    \node [rep,fill=\outputcolor] (output) at (0,10){$y$} ;
    \node [gt, minimum width=1cm] (softmax) at (0,9-0.1) {$\softmax$};
    \draw [arrowfunction] (hidden3) -- (softmax) -- (output); 

\end{tikzpicture}
}
    \caption{A single layer network pretrained on equality computing hierarchical equality.}
    \label{fig:models:premack}
  \end{subfigure}
  \newline
  \begin{subfigure}[t]{1.0\textwidth}
    \centering
    \resizebox{400pt}{!}{%
\begin{tikzpicture}[
    % GLOBAL CFG
    font=\sf \large,
    >=LaTeX,
    % Styles
    rep/.style={% For representations
        rectangle, 
        rounded corners=3mm, 
        draw,
        very thick,
        minimum height =1cm, 
        minimum width=1.61cm
        },
    function/.style={%For functions
        ellipse,
        draw,
        inner sep=1pt
        },
    gt/.style={% For internal inputs
        rectangle,
        draw,
        minimum width=5mm,
        minimum height=4mm,
        inner sep=1pt
        },
    function/.style={
        rectangle,
        draw,
        minimum width=5mm,
        minimum height=4mm,
        inner sep=1pt
        },
    arrowconcat/.style={% Arrows for concatenation
        rounded corners=.25cm,
        dashed,
        thick,
        ->, 
        },
    arrowfunction/.style={% Arrows for concatenation
        rounded corners=.25cm,
        thick,
        ->,       
        }
    ]

%Start drawing the thing...    
    % Draw the cell: 
    \node [rep, fill=\inputcolor] (input1) at (-10,4){$<s>$} ;
    \node [rep, fill=\inputcolor] (input2) at (-6,4){$A$} ;
    \node [rep, fill=\inputcolor] (input3) at (-2,4){$B$} ;
    \node [rep, fill=\inputcolor] (input4) at (2,4){$A$} ;

    \node [rep, fill=\hiddencolor] (hidden0) at (-10,6){$h_0$} ;    
    \node [rep, fill=\hiddencolor] (hidden1) at (-6,6){$h_1$} ;
    \node [rep, fill=\hiddencolor] (hidden2) at (-2,6){$h_2$} ;
    \node [rep, fill=\hiddencolor] (hidden3) at (2,6){$h_3$} ;
    \node [rep, fill=\hiddencolor] (hidden4) at (6,6){$h_4$} ;
    
    \node [gt] (LSTM0) at (-8,6){$\LSTM$} ;
    \node [gt] (LSTM1) at (-4,6){$\LSTM$} ;
    \node [gt] (LSTM2) at (0,6){$\LSTM$} ;
    \node [gt] (LSTM3) at (4,6){$\LSTM$} ;
    
    \node [rep, fill=\outputcolor] (output1) at (-6,8){$y_1$} ;
    \node [rep, fill=\outputcolor] (output2) at (-2,8){$y_2$} ;
    \node [rep, fill=\outputcolor] (output3) at (2,8){$y_3$} ;
    \node [rep, fill=\outputcolor] (output4) at (6,8){$y_4$} ;

    \node [gt] (compare1) at (-6,9.5){$\compare$} ;
    \node [gt] (compare2) at (-2,9.5){$\compare$} ;
    \node [gt] (compare3) at (2,9.5){$\compare$} ;
    \node [gt] (compare4) at (6,9.5){$\compare$} ;
    
    \node [rep, fill=\colorlabel] (label1) at (-6,11){$A$} ;
    \node [rep, fill=\colorlabel] (label2) at (-2,11){$B$} ;
    \node [rep, fill=\colorlabel] (label3) at (2,11){$A$} ;
    \node [rep, fill=\colorlabel] (label4) at (6,11){$</s>$} ;

    \draw [arrowfunction] (label1) -- (compare1); 
    \draw [arrowfunction] (label2) -- (compare2); 
    \draw [arrowfunction] (label3) -- (compare3); 
    \draw [arrowfunction] (label4) -- (compare4); 
    
    \draw [arrowfunction] (output1) -- (compare1); 
    \draw [arrowfunction] (output2) -- (compare2); 
    \draw [arrowfunction] (output3) -- (compare3); 
    \draw [arrowfunction] (output4) -- (compare4); 

    \draw [arrowfunction] (hidden1) -- (output1); 
    \draw [arrowfunction] (hidden2) -- (output2); 
    \draw [arrowfunction] (hidden3) -- (output3); 
    \draw [arrowfunction] (hidden4) -- (output4); 

    \draw [thick] (input1) to[out=0,in=-180, distance=1cm] (LSTM0);
    \draw [thick] (input2) to[out=0,in=-180, distance=1cm] (LSTM1);
    \draw [thick] (input3) to[out=0,in=-180, distance=1cm] (LSTM2);
    \draw [thick] (input4) to[out=0,in=-180, distance=1cm] (LSTM3);

    \draw [arrowfunction] (hidden0) -- (LSTM0) -- (hidden1); 
    \draw [arrowfunction] (hidden1) -- (LSTM1)-- (hidden2); 
    \draw [arrowfunction] (hidden2) -- (LSTM2)-- (hidden3); 
    \draw [arrowfunction] (hidden3) -- (LSTM3)-- (hidden4); 
    
    \draw[->,dashed,thick] (label1) to[out=0,in=-90, distance=1.99cm] (input2);
    \draw[->,dashed,thick] (label2) to[out=0,in=-90, distance=1.99cm] (input3);
    \draw[->,dashed,thick] (label3) to[out=0,in=-90, distance=1.99cm] (input4);


\end{tikzpicture}
}
    \caption{A recursive LSTM network producing ABA sequences.}
    \label{fig:reps:sequence}
  \end{subfigure}
  \caption{The models for our three tasks.}
  \label{fig:models}
\end{figure*}